\subsection{Radius}

The radius of a neutron star is typically on the order of 10 to 15 kilometers \citeauthorandyear{kutschera1998neutron}, making them incredibly compact objects. To put this into perspective, neutron stars have a radius that is about 1,000 times smaller than the radius of the Earth and only about 10 kilometers across.\\

The exact radius of a neutron star depends on its mass, composition, and equation of state, which describes the relationship between density, pressure, and other properties of the stellar matter. Neutron stars with higher masses tend to have slightly smaller radii due to the stronger gravitational forces. A maximum limit for the radius of neutron stars have been found out to be between 10.4 - 11.9 km \citeauthorandyear{Capano2020}\\

The determination of the radius of a neutron star is a challenging task and requires sophisticated modeling and observational techniques. Indirect methods, such as analyzing the X-ray spectra and thermal emission from the surface of neutron stars, can provide valuable insights into their radius. Additionally, multi-wavelength observations, such as the study of gravitational waves from neutron star mergers, can help constrain the radius and other properties of neutron stars.\\

It is important to note that the concept of the ``surface" of a neutron star is not well-defined, as the outer layers of a neutron star can consist of a solid crust, a liquid layer, and a gaseous atmosphere. The structure and composition of these layers can vary depending on the mass and age of the neutron star \citeauthorandyear{kayanikhoo2023maximum}.\\

\subsection{Magnetic Fields}

Magnetic fields in neutron stars are incredibly strong, making them some of the most powerful magnetic fields known in the universe. Neutron stars possess magnetic fields ranging from approximately \num{e8} to \num{e15} Gauss (G), with the higher end of the range representing magnetars, a special class of highly magnetized neutron stars. The origin and strength of the magnetic fields in neutron stars are still active areas of research, but there are several proposed mechanisms for their formation. One theory suggests that the magnetic fields of neutron stars are remnants of the magnetic fields of the progenitor stars. As the massive star undergoes a supernova explosion and collapses, its magnetic field gets amplified due to the conservation of magnetic flux, leading to the formation of a highly magnetized neutron star.

Another possibility is the "dynamo mechanism," which involves the generation and amplification of magnetic fields through various fluid and plasma processes occurring in the core of the neutron star. These processes can be driven by the rapid rotation of the neutron star and the convection of charged particles in its interior.\citeauthorandyear{duncan_thompson}

\subsection{Density}
The density of neutron stars is one of their most remarkable and defining characteristics. Neutron stars are incredibly dense, packing a tremendous amount of mass into a relatively small volume. The density of a neutron star is typically on the order of \num{e17} to \num{e18} kilograms per cubic meter, which is several orders of magnitude greater than the density of atomic nuclei.\\

To put this into perspective, the density of an atomic nucleus is around \num{e17} kilograms per cubic meter, and the density of ordinary matter on Earth is approximately \num{e3} kilograms per cubic meter. Thus, the matter in a neutron star is packed so tightly that it is denser than any known form of matter on Earth.\\

The high density of neutron stars arises from the tremendous gravitational forces present within these compact stellar remnants. The gravitational collapse during the formation of a neutron star compresses the matter to such an extent that the electrons and protons in atomic nuclei combine to form tightly packed neutrons. This neutron-rich matter becomes so dense that it surpasses the densities observed in atomic nuclei.\\

The exact density of a neutron star depends on several factors, including its mass and radius. Typically, neutron stars have masses around 1.4 times that of the Sun, but their radii are only about 10 to 15 kilometers. This compact size, combined with the substantial mass, results in extreme densities.

\subsection{Escape Velocity}
The escape velocity of a neutron star depends primarily on its mass and radius. The mass of a typical neutron star is approximately 1.4 times the mass of the Sun, while its radius is on the order of 10 to 15 kilometers. These values lead to escape velocities of the same magnitude.\\

\noindent
Assuming a mass of 1.4 solar masses (M $\approx \num{2.8e30} kg$) and a radius of 10 kilometers (R $\approx$ 10,000 meters), the escape velocity can be calculated as follows:

\begin{equation*}
    v = \sqrt{\frac{2 M G}{r}}
    = \sqrt{\frac{2 \times \num{2.8e30} \times \num{6.67430e-11}}{10000}}
    = \SI{6.7e7}{\metre/\second}
\end{equation*}

This escape velocity is approximately 0.22 times the speed of light, emphasizing the immense gravitational forces at play near neutron stars. It implies that any object, including light itself, would need to exceed this velocity to escape the gravitational pull and move away from the neutron star indefinitely.\\

\newpage
The high escape velocity of neutron stars has significant implications for their observational properties. It means that the intense gravitational pull can cause substantial gravitational redshift, where the light emitted from the neutron star's surface is stretched to longer wavelengths due to the strong gravitational field. Additionally, it leads to time dilation effects, where time appears to move slower for observers near the neutron star compared to distant observers.\\

Understanding the escape velocities of neutron stars helps us appreciate the immense gravitational forces and extreme conditions present in these objects. The escape velocity acts as a crucial parameter in comprehending the physics and observational aspects associated with neutron stars.

\subsection{Surface Temperature}

The surface temperature of neutron stars varies depending on their age, composition, and thermal evolution. Newly formed neutron stars, known as "hot" neutron stars, can have surface temperatures in the range of millions to tens of millions of degrees Kelvin. However, over time, these temperatures decrease as the neutron star cools and loses thermal energy.\\

The high initial temperatures of neutron stars are a result of their formation processes, particularly the intense energy released during the supernova explosion. During the core collapse and subsequent rebound, a significant amount of energy is generated, contributing to the initial high temperatures of the nascent neutron star.\\

As the neutron star ages, it gradually cools down through a combination of neutrino emission and thermal radiation from its surface. Neutrino cooling dominates in the early stages, as neutrinos, which interact weakly with matter, carry away thermal energy from the star's interior. However, as the neutron star evolves and its interior temperature drops, the rate of neutrino emission decreases, and the surface cooling becomes increasingly important.\\

The cooling process is governed by various factors, such as the composition of the neutron star's crust and core, the presence of magnetic fields, and the rate of energy transfer from the interior to the surface. These factors influence the cooling rate and the final equilibrium temperature reached by the neutron star's surface.\\

Observationally, the surface temperatures of older, "cooled" neutron stars generally range from thousands to hundreds of thousands of degrees Kelvin. These cooler temperatures make them primarily emit thermal X-rays, which are often detected by X-ray telescopes. The exact surface temperature of a neutron star depends on its mass, age, and the efficiency of thermal energy transport mechanisms.\\

It is worth noting that not all neutron stars have the same surface temperature. Neutron stars can exhibit a wide range of temperatures depending on their specific characteristics, such as whether they are isolated or in binary systems, and whether they are accreting matter from a companion star. Factors such as accretion processes, magnetic activity, and the presence of a surrounding atmosphere can significantly influence the observed surface temperature.\\

Determining the precise surface temperature of a neutron star requires detailed observations and sophisticated modeling techniques. Measurements of the thermal radiation emitted by neutron stars across the electromagnetic spectrum, particularly in X-rays, provide valuable insights into their cooling processes and the underlying physics of these exotic objects.\citeauthorandyear{kayanikhoo2023maximum}

\subsection{Rotation Speed}

Neutron stars are known for their incredibly rapid rotation speeds, often referred to as their spin or rotation periods. These rotation speeds can range from milliseconds to several seconds, depending on the individual neutron star. The rapid rotation of neutron stars is a consequence of the conservation of angular momentum during their formation.\\

When a massive star collapses, its core shrinks and conserves its angular momentum, resulting in a drastic increase in rotation speed due to the reduction in size. This phenomenon is similar to an ice skater spinning faster when they pull in their arms.\\

The fastest rotating neutron stars are known as pulsars. Pulsars are highly magnetized neutron stars that emit beams of electromagnetic radiation from their magnetic poles. These beams of radiation are observed as regular pulses of radiation as the neutron star rotates and the beams sweep across our line of sight. Pulsars can rotate hundreds of times per second, with the fastest known pulsar, named PSR J1748-2446ad, rotating at approximately 716 times per second.\\

The rotation speed of a neutron star can be measured using various observational techniques. Pulsar timing is a common method where precise measurements of the arrival times of the pulsar's pulses are made over an extended period. By tracking the timing of these pulses, scientists can determine the rotation period of the neutron star with high precision.\\

The rapid rotation of neutron stars has several important implications. The extreme centrifugal forces due to their fast rotation cause the neutron stars to be oblate or slightly flattened at the poles. Additionally, the high rotational energy of neutron stars can power various astrophysical phenomena, such as the emission of high-energy radiation, the generation of strong magnetic fields, and the production of jets of relativistic particles.\\

In some cases, neutron stars can experience changes in their rotation speed over time. This phenomenon, known as ``glitching", occurs when the neutron star's crust undergoes sudden readjustments, causing a brief increase in rotation speed. Glitches are thought to be caused by the transfer of angular momentum from the superfluid interior of the neutron star to the rigid crust.\\

Understanding the rotation speeds of neutron stars provides valuable insights into their internal structure, magnetic fields, and evolution. It allows scientists to study extreme physical phenomena under the influence of strong gravity and magnetic fields, shedding light on fundamental physics and astrophysical processes.\citeauthorandyear{Heger_2003}