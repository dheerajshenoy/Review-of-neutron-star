\newpage
\subsubsection{Black Widow Pulsars}

Black widow pulsars (BWPs) are a special class of binary millisecond pulsars (MSPs). Their orbital periods range from 2-20 hours \cite{Formation_BWP_Chen_2013} and  the defining feature of BWPs is theirlow mass companions, with masses much less than 0.1 times of mass of the sun \cite{Bochenek_2015}. Even though BWPs have very small companions, the ionized gas from the companions often eclipses the pulsar. This occurs because the pulsarwind blows stellar material away from the companion, but leaves plasma in the system, creating a screen for the pulsar’s radio emission\cite{Bochenek_2015}.\\

With rotation periods of 10 milliseconds or less, the BWPs spin at very high speeds, upto 43,000 rpm. Today, more than 300 of these millisecond pulsars have been catalogued. Young neutron stars usually appear in isolation, more than half of them have a stellar partner, suggesting that interactions with a normal star can rejuvenate old, slow neutron stars.\\

The intense radiation emitted by the pulsar impacts the companion star, causing it to be heated and partially ablated. This phenomenon earned the Black Widow Pulsar its name, as it metaphorically "consumes" or "destroys" its companion star like a black widow spider consuming its mate. The interaction between the pulsar and its companion has dramatic effects on the companion star. The intense radiation and stellar winds from the pulsar strip away the outer layers of the companion star, creating a trailing gas tail. The process of ablation can lead to the erosion of the companion star's atmosphere and even result in the eventual complete evaporation of the star.

