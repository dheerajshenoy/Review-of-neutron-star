
When the nuclear fuel in a star is exhausted, the gaseous clouds start to collapse under the gravitational pull and explodes in a supernova event. After supernova, only a fraction of the mass is left, and the final product might be a white dwarf, a neutron star or a black hole, depending on the mass of the progenitor. If the collapsing core is about 8 - 25 \mdot, where \mdot is the solar mass $\approx \num{2e33}$ g, the electrons and protons in the gaseous cloud get converted to neutrons under high pressure to form a neutron star. Further collapse is prevented by neutron degeneracy pressure, a phenomenon described by the Pauli Exclusion Principle. If the mass of the neutron star formed is in excess of 2 - 3 \mdot the mass is greater than the above mentioned limit then it leads to the formation of a black hole.\\

Neutron stars are characterized by their high densities, typically around 1.4 to 2 times the density of atomic nuclei. This extreme density is responsible for their unique properties and behaviors. The neutrons in the core are packed so tightly that they interact strongly with each other, giving rise to a state of matter called neutronium. Neutron stars possess incredibly strong gravitational fields, which warp the surrounding spacetime and create gravitational wells.\\

The formation of a neutron star is a violent and energetic process, releasing an enormous amount of energy in the form of neutrinos and a supernova explosion. These explosions contribute to the dispersal of heavy elements and enrich the interstellar medium with the products of nucleosynthesis.\\

In recent years, the detection of gravitational waves from neutron star mergers has provided additional insights into their formation. When two neutron stars in a binary system spiral inward due to the emission of gravitational waves, they eventually collide and merge. This merger event releases a tremendous amount of energy in the form of gravitational waves, gamma-ray bursts, and other electromagnetic radiation, leading to the formation of highly massive neutron stars or even black holes.\\

The formation of neutron stars is a captivating process that highlights the incredible forces and extreme conditions present in the universe. The study of neutron star formation not only expands our knowledge of stellar evolution but also provides a deeper understanding of the fundamental physics governing matter at extreme densities and temperatures.\\

\begin{figure}[h]
\centering
\includegraphics[height=0.9\textwidth, width=0.9\textwidth]{images/Neutronstarsimple.png}
\caption{\small Simplified representation of the formation of neutron stars credits; Bedrock Person; 22 June 2017}
\end{figure}