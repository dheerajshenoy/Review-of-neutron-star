% SECTION : ABSTRACT
\setcounter{page}{1}

    
\begin{center}
    \large{\textbf{Abstract}}
\end{center}

\noindent
Most dense star in the universe is neutron star. We discuss here briefly the formation of neutron stars and their types, and then gravitational wave analysis. We discuss the challenges that are involved in gravitational wave analysis of neutron stars particularly magnetars, and how it can be analysed to understand the processes involved in the formation of these peculiar stars. We discuss the sensitivity of the current gravitational wave detectors. The magnetic fields of neutron stars are a crucial aspect of their nature, and their origin and influence are explored in detail. The paper discusses various mechanisms for the generation and amplification of magnetic fields in neutron stars, including the role of progenitor stars and dynamo processes. The immense strength of neutron star magnetic fields and their impact on the star's structure, radiation emission, and surrounding environment are also addressed. We cover the detection of neutron stars through pulsar observations, X-ray and gamma-ray emission, and gravitational wave signals. The paper also discusses the exploration of highly magnetized neutron stars known as magnetars and their unique phenomena, such as bursts and flares.\\

\noindent
\textbf{Keywords}: Neutron stars, Magnetars, Pulsars, Black Widow pulsars, Gravitational Wave, Relativity, Gamma Ray Burst, Flares, Event Rate, Continuous Gravitational Waves.
