\newpage\\
\section{Conclusion}

We have provided a brief overview of neutron stars, their types, and the processes involved in their formation, analysis of their GW waves in this review study. We have covered Black Widow Pulsars and Magnetars, two unique varieties of neutron stars. We discussed the radius, density, rotational speed, surface temperature, and escape velocity of neutron stars, among other characteristics. We can learn more about the makeup of neutron stars by analysing the gravitational waves that high magnetic neutron stars release, known as Continuous Gravitational Waves. It is still unknown how magnetars and black widow pulsars are formed. Continuous gravitational waves can be used to approximate neutron star parameters, such as their mass, radius etc. which will help us characterise stars as they get closer to becoming one of the different types of neutron stars and understand the structure and makeup of Magnetar and other families of neutron stars. The first continuous observation of gravitational waves would be a significant scientific discovery. However, what we learn as a result of that discovery may be even more astounding.