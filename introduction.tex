% INTRODUCTION

\normalsize

In the vast tapestry of the cosmos, where celestial bodies of various sizes and compositions populate the universe, neutron stars stand out as extraordinary phenomena. These enigmatic objects, born from the violent deaths of massive stars, are the remnants of stellar evolution pushed to the brink of collapse. Neutron stars are captivating not only due to their extreme physical properties but also because they provide a unique window into the fundamental laws of physics operating in the most extreme environments imaginable.\\

In 1934, two astronomers, Baade and Zwicky, in their paper \citeauthorandyear{baade_zwicky} hypothesized the existence of neutron stars as an ultimate fate of any ordinary star once their nuclear fuels are exhausted. This came shortly after the hypothesis of existence of neutron by James Chadwick  \citeauthorandyear{james_chadwick} in 1932. In 1967, Jocelyn Bell, a postgraduate student working under Anthony Hewish, \citeauthorandyear{bell_hewish}, discovered a star that was pulsating, which later came to be known as pulsars, which is a type of neutron star.\\

Neutron stars, despite their name, are not composed solely of neutrons. Rather, they consist of an unimaginably dense and compact core formed predominantly by tightly packed neutrons. This exotic state of matter, known as neutronium, arises under the immense gravitational forces that act upon the collapsing stellar core during the supernova explosion. The resulting neutron star boasts an average mass comparable to that of our Sun but compressed into a sphere with a radius on the order of only a dozen kilometers. Such extraordinary densities give rise to mind-boggling physical phenomena, making neutron stars a playground for scientists to explore the frontiers of physics.\\

\newpage
Neutron stars exhibit a diverse range of fascinating behaviors and emit various forms of electromagnetic radiation across the electromagnetic spectrum. Pulsars, a type of rapidly spinning neutron star, emit regular beams of radio waves, creating a lighthouse-like effect as the beams sweep across our line of sight. These ``cosmic beacons" have become valuable tools for probing the underlying physics of neutron stars and for studying the properties of interstellar matter and magnetic fields.\\

Furthermore, neutron stars are involved in extreme astrophysical events, such as supernova explosions and binary star mergers. These cataclysmic events release vast amounts of energy and generate powerful phenomena like gamma-ray bursts and gravitational waves. By studying the aftermath of such events, scientists can deepen their understanding of the physical processes that govern the universe and gain insights into the formation of heavy elements, like gold and platinum.\\